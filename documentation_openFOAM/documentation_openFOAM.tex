% !TeX document-id = {540a5523-1dc1-449e-ae26-9e8791e51218}

\documentclass[a4paper, 12pt]{article}
%\usepackage[T1]{fontenc}
\usepackage[utf8]{inputenc}
\usepackage[english]{babel}
\usepackage{indentfirst}       %adicionar identação em parágrafos
\usepackage{setspace}          %definir espaçamento entre linhas
\usepackage{xcolor}            %usar letras coloridas
%\usepackage{lmodern}
%\usepackage{float}
%\usepackage{placeins}
%\usepackage{exscale}
%\usepackage{times}

%cuida das coisas de appendice
\usepackage[titletoc,title]{appendix}   % Better appendix


\usepackage[square,numbers]{natbib}
%\usepackage{natbib}              % Citações tipo (nome-ano)
\usepackage{float}

\usepackage{multirow}
\usepackage{pbox}

\usepackage[none]{hyphenat}
\usepackage[makeroom]{cancel}
\usepackage{amsmath}
\usepackage{mathtools}
%\usepackage{amsfonts}
\usepackage{amssymb}
%\usepackage{amsthm}
\usepackage{xfrac}
\usepackage[mathscr]{euscript}

\usepackage{overpic}
\usepackage{subfigure}

%numeracao melhorada
%\usepackage{enumitem}
%\setlist[enumerate]{label=(\arabic*)}

\usepackage[hidelinks]{hyperref}    %idem ao de cima, mas sem o quadradinho feio nas páginas...

\usepackage{mathtools}

\usepackage{titlesec}
\newcommand{\sectionbreak}{\clearpage}
\setcounter{secnumdepth}{3}

\usepackage{pythonhighlight}

\usepackage{tikz}
\usepackage{tikz-3dplot}

\usepackage{helvet}
% !TeX TXS-program:compile = txs:///pdflatex/[--shell-escape]
\usepackage{minted}

%%%%%%%%%%%%%%%%%%%%%%%%%%%%%%%%%%%%%%%%%%%%%%%%%%%%%%%%%%%%%%%%%%

\numberwithin{equation}{section}
\onehalfspacing         % Espaçamento de 1.5
\raggedbottom
\sloppy

\newcommand{\code}{\fontfamily{pcr}\selectfont}

%colorido
\newcommand\Ccancel[2][red]{\renewcommand\CancelColor{\color{#1}}\cancel{#2}}
\newcommand{\blue}[1]{\textcolor{blue}{#1}}
\newcommand{\red}[1]{\textcolor{red}{#1}}


%line spacing table
\renewcommand{\arraystretch}{1.0}


%code by Heiko Oberdiek
\makeatletter
%Roman counter
\newcounter{roem}
\renewcommand{\theroem}{\roman{roem}}

% save the original counter
\newcommand{\c@org@eq}{}
\let\c@org@eq\c@equation
\newcommand{\org@theeq}{}
\let\org@theeq\theequation

%\setroem sets roman counting
\newcommand{\setroem}{
	\let\c@equation\c@roem
	\let\theequation\theroem}

%\setarab the arabic counting
\newcommand{\setarab}{
	\let\c@equation\c@org@eq
	\let\theequation\org@theeq}
\makeatother


%\renewcommand*\descriptionlabel[1]{\hspace\leftmargin$#1$}


%organizando margens
\voffset=-2cm
\setlength{\textheight}{24cm}
\hoffset=-1.6cm
\setlength{\textwidth}{17cm}

\title{OpenFOAM}
\author{Tulio Rodarte Ricciardi}
\date{\today}

\begin{document}

	\maketitle
	\tableofcontents


\section{TODO}
    - transport models

    \href{https://www.cfd-online.com/Forums/openfoam-community-contributions/103227-multi-species-mass-transport-library-update-6.html}{\blue{library up to openfoam 6}}

    \href{https://github.com/ZSHtju/reactingDNS_OpenFOAM}{\blue{OpenFOAM-8 ``working'' version}}

    - sponge (?)

    - constraint fields:
    \href{https://www.openfoam.com/documentation/guides/latest/doc/guide-fvoptions-constraints-fixed-value.html}{\blue{constraint}}

\section{Building}

	\subsection{Pre-reqs}

        \paragraph{Setup environment}
            \begin{itemize}
            \item DANE: \\
            {\code module unload intel-classic/2021.6.0-magic mvapich2/2.3.7} \\
            {\code module load gcc/11.2.1 openmpi/4.1.2} \\
            check that cmake exists

    	    \item Campus Cluster: \\
    	    {\code module load cmake/3.26.3} \\
    	    {\code module unload gcc/11.2.0} \\
    	    {\code module load openmpi/.4.1.4-gcc-12.2.0} \\
    	    {\code \red{src/OpenFOAM/db/IOstreams/hashes/}}: see \href{https://github.com/OpenFOAM/OpenFOAM-9/commit/b0c15bebd37142f3902901ed5e9a60e33ed456eb}{\bf{\blue{link}}}

            \item Local installation: \\
    		{\code sudo apt install flex}
            \end{itemize}

        \paragraph{openMPI \\}
            Install \href{https://docs.open-mpi.org/en/v5.0.x/installing-open-mpi/quickstart.html}{\blue{openMPI}}

        \paragraph{Cantera \\}
            {\code python3 -m pip install cantera}

        \paragraph{gmsh \\}

            {\code cd /g/g92/$<$your-username$>$}

            {\code wget https://gmsh.info/bin/Linux/gmsh-4.13.1-Linux64.tgz}

            {\code tar zxv gmsh-4.13.1-Linux64.tgz}

            {\code export PATH="\$PATH:/g/g92/$<$your-username$>$/gmsh-4.13.1-Linux64/bin"}

    \subsection{openFOAM}
        \begin{enumerate}
        \item Download openFOAM: \\
        OFversion=... \\
        wget -O - http://dl.openfoam.org/third-party/\$\{OFversion\} $|$ tar zxv \\
        mv ThirdParty-\$\{OFversion\}-version-\$\{OFversion\} ThirdParty-\$\{OFversion\} \\
        wget -O - http://dl.openfoam.org/source/\$\{OFversion\} $|$ tar zxv \\
        mv OpenFOAM-\$\{OFversion\}-version-\$\{OFversion\} OpenFOAM-\$\{OFversion\} \\

        \item Build openFOAM: \\
        source $<$path$>$/OpenFOAM-\$\{OFversion\}/etc/bashrc \\
        ./Allwmake $<$-j np$>$

        \item Add to .bashrc for future use (or as an alias): \\
        source $<$path$>$/OpenFOAM-\$\{OFversion\}/etc/bashrc
        \end{enumerate}

    \subsection{Adding a new solver}

        \subsubsection{reactingDNS}

            \begin{enumerate}
            \item Needs OpenFOAM 8, so follow the previous step with {\code OFversion=8} but do not build it. Just download everything.

            \item Copy the folder {\code reactingDNS} from {\code $<$github$>$/applications/solvers/combustion} to {\code $<$path$>$/OpenFOAM-8/applications/solvers/combustion}

            \item Sometimes, this file gives issue. I noticed that, independent on the version, it always works fine with the updated file: \\
            {\code cd $<$path$>$/OpenFOAM-8/src/OpenFOAM/db/IOstreams/hashes/} \\
            {\code mv OSHA1stream.H \_OSHA1stream.H} \\
            {\code mv $<$github$>$/OpenFOAM-8/src/OpenFOAM/db/IOstreams/hashes/ .}

            \item Source and build OpenFOAM 8

            \item Use the files in {\code $<$github-folder$>$/reactingDNS} to run the simulation
            \end{enumerate}

    \subsection{PATO}
        \begin{itemize}
        \item Build PATO: \\
        wget https://github.com/nasa/pato/archive/refs/tags/3.1.tar.gz \\
        tar -zxvf 3.1.tar.gz \\
        export PATO\_DIR=\$(pwd)/PATO/pato-3.1 \\
        source $<$path$>$/OpenFOAM-7/etc/bashrc \\
        source \$PATO\_DIR/bashrc \\
        cd \$PATO\_DIR/src/thirdParty/mutation++ \\
        See instruction in \ref{sec:mutation} for {\code mutation++} \\
        ln -s \$(pwd)/thirdparty/eigen/Eigen/ ./install/include \\
        cd \$PATO\_DIR \\
        ./Allwmake

        \item Alias in .bashrc
        \end{itemize}

        %PS: it may have some issues with mutation++: installing the thirdParty version manually seems easier; also need to go to \textit{install/include} and manually link the \textit{Eigen} library in their thirdParty folder: it struggles with the Eigen package, so just do it by hand...

        %PS: check the \textit{sources} and \textit{exports}. If it doesn't find \textit{wmkdep} is because of export and sources. %I also rebuild the ``ThirdParty-7'' from openfoam before it worked...

    \subsection{Mutation++ \\}
    \label{sec:mutation}

        If outside PATO: \\
        git clone https://github.com/mutationpp/Mutationpp.git

        Build Mutation: \\
        % mkdir build \\
        % module load cmake/3.12.0 \\
        cmake -DCMAKE\_INSTALL\_PREFIX:PATH=$<$path$>$/install \\
        make -j 4 install \\

        If installing inside PATO, it is done.

        Else, if stand-alone, add to .bashrc: \\
        export MPP\_DIRECTORY=$<$path$>$ \\
        export MPP\_DATA\_DIRECTORY=\$MPP\_DIRECTORY/data \\
        export PATH=\$MPP\_DIRECTORY/install/bin:\$PATH \\
        export LD\_LIBRARY\_PATH=\$MPP\_DIRECTORY/install/lib64:\$LD\_LIBRARY\_PATH

        Can check just in case: \\
        checkmix air\_11

    \subsection{TODO: (ignore this)}
    \paragraph{swak4Foam \\}

        \href{https://www.cfd-online.com/Forums/openfoam-community-contributions/222909-swak4foam-installation-openfoam-7-a.html}{\blue{swak4Foam}}

\section{Environment configuration and running}

    \subsection{Dane:}
        Before running OpenFOAM: \\
        {\code module unload intel-classic/2021.6.0-magic mvapich2/2.3.7} \\
        {\code module load gcc/11.2.1 openmpi/4.1.2} \\
        {\code source /g/g92/$<$your-username$>$/openFOAM/OpenFOAM-$X$/etc/bashrc} \\
        {\code cd $<$path/to/the/case$>$}

        A whole bunch of configuration (i.e., case-dependent) \\
        {\code sh setFields.sh}

        Run with this (add to the queue system file): \\
        {\code mpirun -np 3 buoyantReactingFoam -parallel}

%    \paragraph{Sourcing}
%    \begin{itemize}
%    \item Add the following lines to a file in your home (something like {\code ~/.init\_of7.sh}): \\
%    {\code module unload intel-classic/2021.6.0-magic mvapich2/2.3.7} \\
%    {\code module load gcc/11.2.1 openmpi/4.1.2} \\
%    {\code source /g/g92/$<$your-username$>$/openFOAM/OpenFOAM-$X$/etc/bashrc}
%
%    \item Then add to {\code .bashrc} \\
%    {\code alias init\_of7="sh ~/.init\_of$X$.sh"}
%
%    \item Then, just type {\code init\_of$X$} before running OpenFOAM
%    \end{itemize}

\section{Setup}

    \subsection{Basic steps}

        Steps for running openFOAM simulations: \\
        {\code gmshToFoam mesh.msh} \\
        {\code $>$ update constant/boundary with {\code wedge} if axisymmetric} \\
        {\code $>$ update constant/boundary with {\code wall} may be necessary} \\
        {\code create 0 folder with initial and boundary conditions} \\
        {\code $>$ setFields -region flow for non-uniform IC} \\
        {\code splitMeshRegions -cellZones -overwrite} \\
        {\code $>$ decomposePar -allRegions if running in parallel}

    \subsection{Mesh generation}
    \label{subsec:mesh_gen}

        {\code gmshToFoam mesh.msh}

        Quads that touch the boundaries with a single node gives trouble and make a new ``boundary'' (defaultFaces). This may lead to issues on the simulation. So far, the best I could do was to run mixed meshes, with quads/hexes on well behaved regions (transfinite) and triangles everywhere else.

        When creating a mesh with gmsh, you can specify multiple volumes, without splitting, combining or prescribing anything to their boundaries. This will be treated as different {\code cellZone}, which can then be used for unique purposes.

        \subsubsection{blockMesh}

            {\code paraFoam -block}

        \subsubsection{axisymmetry}

            Using {\code writeFormat binary;} in {\code controlDict} helps depending on openFOAM version

            When meshing, add two separate boundaries: front and back

            Add {\code wedge} as BC for all variables in front and back

            in {\code polyMesh/boundary}, modify from {\code patch} to {\code wedge} where necessary

            \href{https://www.youtube.com/watch?v=MUGnlYncqck}{\blue{link}}

            \begin{center}
            \fbox{\begin{minipage}[t]{0.8\textwidth}
            \centering
            PS: I see non-zero Z-velocity on the wedge faces.
            \end{minipage}}
            \end{center}

    \subsection{Initial conditions}

        Units: [kg m s K mol A cd]

        In an incompressible solver, N-S equation is divided by a uniform density rho. This causes (1) the dimension of pressure of [0 2 -2 0 0 0 0] and (2) the kinematic viscosity $\nu$ in Laplacian term. In an incompressible solver, pressure is assumed to be relative. The atmosphere will be 0 usually.

        In a compressible solver, N-S equation is not divided by density. So, the dimension of pressure is [1 -1 -2 0 0 0 0] as usual. The dynamic viscosity $\mu$ appears in Laplacian term. In a compressible solver, the absolute pressure must be provided in $p$ file because the value of pressure will be used to calculate other physical properties. The atmosphere will be O(1e5) usually.

    \subsection{Boundary conditions}

        \subsubsection{Variable BCs}

            custom boundary conditions:

            \href{
            https://www.youtube.com/watch?v=3zcTxGHxuok
            }{\blue{link}}

            \href{
            https://www.cfd-online.com/Forums/openfoam-solving/119239-using-codedfixedvalue-apply-totalpressure-boundary-condition.html
            }{\blue{link}}

        \subsubsection{Pressure BC + gravity}

            \textbf{Walls} :: use {\code fixedFluxPressure} or {\code fixedFluxExtrapolatedPressure} (instead of {\code zeroGradient}) or it will give issues.

            %in cases where {\code p\_rgh} is solved, it seems that we have to apply the BC as {\code fixedValue} (or {\code totalPressure}) and {\code calculated} for p; for u, to avoid reversed flow and make the simulation more robust, the type is {\code pressureInletOutletVelocity} and {\code value uniform (0 0 0)}, where it behaves as zero-gradient condition when outlet and prescribed flow when inlet.

            \textbf{Outlet} :: To satisfy the characteristics and avoid under/over-specification, gotta prescribe U and T (potentially Y) when there is reversed flow in the outlet; pressure is extrapolated by using a zero-gradient BC. For outflow, the internal properties are extrapolated but pressure is prescribed.

            See \href{
            https://doc.openfoam.com/2306/tools/processing/boundary-conditions/rtm/derived/inletOutlet/inletOutlet/
            }{\blue{inletOutlet}}

            See \href{
            https://doc.openfoam.com/2306/tools/processing/boundary-conditions/rtm/derived/inletOutlet/fixedMeanOutletInlet/
            }{\blue{fixedMeanOutletInlet}}

            \textbf{Far-field} :: Potentially, using the non-reflective {\code waveTransmissive} may help suppress some weird(-but-small) behavior (which may be waves reflection). This would be useful in the sides/bottom, but \textbf{may not} be robust in the outlet. Additionally, the extra source terms (damping and isentropic) can be used in the bottom and right to damp velocity fluctuations and the initial vortex for increased robustness.

            {\setstretch{1.0}
            \begin{minted}{c}
            farfield
            {
                type            waveTransmissive;
                gamma           1.4;
                fieldInf        101325.0;
                lInf            0.1;
                value           uniform 101325.0;
            }
            \end{minted}
            }

        \subsubsection{time-dependent BCs}

            Linear interpolation from t1 to t2

            {\setstretch{1.0}
            \begin{minted}{c}
            inlet
            {
                type uniformFixedValue;
                uniformValue table
                (
                (0.0 (0 0.02 0))  // t1 (Ux_1 Uy_1 Uz_1)
                (2.5 (0 0.10 0))  // t2 (Ux_2 Uy_2 Uz_2)
                );
            }
            \end{minted}
            }

        \subsubsection{Robin BC}

            https://www.cfd-online.com/Forums/openfoam/219113-how-access-boundary-conditions-species-reaction.html

            $---------------------------------$

            https://foamingtime2.wordpress.com/wp-content/uploads/2017/07/convection-bc\_1.pdf

            https://www.cfd-online.com/Forums/openfoam-pre-processing/225340-mixed-boundary-condition.html

            https://www.cfd-online.com/Forums/openfoam/238186-implementation-robin-boundary-condition-openfoam.html

            https://www.cfd-online.com/Forums/openfoam-solving/69464-robin-b-c-mixed-b-c.html

            https://www.cfd-online.com/Forums/openfoam-solving/83194-mixed-boundary-condition.html

            https://www.cfd-online.com/Forums/openfoam-pre-processing/74593-mixed-bc-heat-transfer-laplacianfoam.html\#post295694

        \subsubsection{change of BCs}

            \href{https://www.cfd-online.com/Forums/openfoam-solving/235527-how-change-boundary-condition-after-specific-timein-parallel-solver.html}{\blue{change BC}}

            foamGetDict

    \subsection{Chemistry}

        \subsubsection{mechanisms}

            need {\code yaml2ck} from Cantera

            \$$>>$ {\code chemkinToFoam uiuc\_20sp.dat thermo\_uiuc\_20sp.dat ... \\ transportProperties ../constant/reactions ... \\ ../constant/thermo.compressibleGas}

            \begin{center}
            \fbox{\begin{minipage}[t]{0.8\textwidth}
            \centering
            \textbf{This may still present issues with chemkin format (e.g., spacing) and it may be necessary to fix by hand}
            \end{minipage}}
            \end{center}

        \subsubsection{possible mechanism simplification}

            see Sandia flame example in reactingFoam $>$ RAS

    \subsection{Radiation}

        {\code system/flow/fvOptions}

    {\setstretch{1.0}
    \begin{minted}{c}
    radiation
    {
        type            radiation;
        libs            ("libradiationModels.so");
    }
    \end{minted}
    }

    \subsubsection{viewFactor}

   		{\code faceAgglomerate -region flow}

   		{\code viewFactorsGen -region flow}


    {\setstretch{1.0}
	\begin{minted}{c}
    flow_to_solidLeft
    {
        type           greyDiffusiveRadiationViewFactor;
        qro            uniform 0;
        emissivityMode solidRadiation;
        value          uniform 0;
        emissivity     uniform 1.0; // Emissivity of the wall
    }
	\end{minted}
	}

    \subsubsection{P1 model}

	        {\code constant file}

	        {\code fvSolution}

	        {\code G file}

	        \href{https://github.com/Cantera/cantera/pull/965/files}{\blue{Species coefficients}}

	        https://boyaowang.github.io/boyaowang\_OpenFOAM.github.io/2020/09/16/P1\_model/

	        https://www.afs.enea.it/project/neptunius/docs/fluent/html/th/node112.htm

	        https://www.cfd-online.com/Forums/openfoam-solving/219601-adding-radiations-chtmultiregionsimplefoam.html

	        https://www.cfd-online.com/Forums/openfoam/78052-boundary-condition-p1-marshakradiation.html

	        https://www.cfd-online.com/Forums/openfoam-programming-development/135502-understanding-marshak-boundary-condition-radiation.html\#post696511

	        https://www.cfd-online.com/Forums/openfoam/183686-radiation-boundary-conditions-flow-through-boundaries-openfoam.html\#post696795

	        https://www.cfd-online.com/Forums/openfoam-solving/240250-p1-model-no-participating-media-heat-fluxes-not-matching.html

	        https://www.cfd-online.com/Forums/openfoam-solving/216879-radiation-models-general-p1-implementation.html

    \subsection{System}

    \subsubsection{fvSchemes}

        \href{https://www.openfoam.com/documentation/user-guide/6-solving/6.2-numerical-schemes}{\blue{fvSchemes}}

        \href{https://doc.cfd.direct/openfoam/user-guide-v7/fvschemes}{\blue{fvSchemes}}

        \href{https://www.wolfdynamics.com/training/introOF8/supplement_tipsandtricks.pdf}{\blue{fvSchemes}}

    \subsubsection{fvSolution}

        \href{https://doc.cfd.direct/openfoam/user-guide-v11/fvsolution}{\blue{link}}

        \paragraph{PISO:\\}

            \href{https://openfoamwiki.net/index.php/PISO}{\blue{link}}

            \href{https://openfoamwiki.net/index.php/OpenFOAM_guide/The_PISO_algorithm_in_OpenFOAM}{\blue{link}}

            \href{https://openfoamwiki.net/index.php/OpenFOAM_guide/The_PIMPLE_algorithm_in_OpenFOAM}{\blue{link}}

            \href{https://www.tfd.chalmers.se/~hani/kurser/OS_CFD_2022/lectureNotes/PIMPLE.pdf}{\blue{link}}

            \href{https://forum.freecad.org/viewtopic.php?style=4&t=31782}{\blue{link}}

            \href{https://doc.cfd.direct/notes/cfd-general-principles/the-pimple-algorithm}{\blue{link}}

            \href{https://doc.cfd.direct/notes/cfd-general-principles/steady-state-solution#x139-18400012}{\blue{link}}

            \href{https://doc.cfd.direct/notes/cfd-general-principles/transient-solution#x150-19700019}{\blue{link}}

            \href{https://www.simscale.com/forum/t/cfd-pimple-algorithm/81418}{\blue{link}}

        \paragraph{convergence:\\}
            In some case, if tolerances are too big, the solution may look a bit off, with pressure oscillations and spurious velocity fields

            Also, it may happen that the flow won't evolve if it is ``converged'' due to a high tolerance, with 0 iterations in the system solvers

            In both cases, just modify the {\code fvSolution} files to have smaller tolerances.

        \paragraph{PIMPLE:\\}

            Looks like 2 (or more) {\code nCorrectors} loops are key to not screw up pressure-velocity coupling. Additionally, {\code nNonOrthogonalCorrectors} may help in regions where the mesh is not properly orthogonal. Issues in this regard were observed when solving a finer mesh. Previously, on a coarser mesh, {\code nCorrectors = 1} and {\code nNonOrthogonalCorrectors = 0} were used just fine.

    \subsection{fvOptions}

    \begin{minipage}{\textwidth}
    {\setstretch{1.0}
    \begin{minted}{c}
    limitT
    {
        type            limitTemperature;
        active          yes;

        selectionMode   all;
        min             200;
        max             2400;
    }
    \end{minted}
    }
    \end{minipage}

    \begin{minipage}{\textwidth}
    \vspace{12pt}
    \href{
    https://github.com/OpenFOAM/OpenFOAM-12/blob/master/src/waves/fvModels/verticalDamping/verticalDamping.C
    }{\blue{verticalDamping.C}}

    {\setstretch{1.0}
    \begin{minted}{c}
    verticalDamping
    {
        type            verticalDamping;

        selectionMode   all;

        origin          (0 0.35 0);  // x y z
        direction       (0 1 0);

        scale  // still unclear what it exactly means
        {
            type        halfCosineRamp;
            start       0;
            duration    .05;
        }

        lambda          [0 0 -1 0 0 0 0] 1000; // Damping coefficient

        timeStart       0;
        duration        5; // time

        writeForceFields true;
    }
    \end{minted}
    }
    \end{minipage}

    \begin{minipage}{\textwidth}
    \vspace{12pt}
    {\setstretch{1.0}
    \begin{minted}{c}
    fixedTemperature
    {
        type            fixedTemperatureConstraint;

        selectionMode   cellZone;
        cellZone        porosity;
        mode            uniform;
        temperature     300;
    }
    \end{minted}
    }
    \end{minipage}

    \begin{minipage}{\textwidth}
    \vspace{12pt}
    {\setstretch{1.0}
    \begin{minted}{c}
    fixedValue
    {
        type            scalarFixedValueConstraint;
        active          yes;

        selectionMode   cellZone;
        cellZone        porosity;
        fieldValues
        {
            mu          1e-5;
        }
    }
    \end{minted}
    }
    \end{minipage}

    \begin{minipage}{\textwidth}
    \vspace{12pt}
    {\setstretch{1.0}
    \begin{minted}{c}
    fixedValue
    {
        type            vectorFixedValueConstraint;
        active          yes;

        selectionMode   cellZone;
        cellZone        porosity;
        fieldValues
        {
            U           (1.0 0.0 0.0);
        }
    }
    \end{minted}
    }
    \end{minipage}

    \subsection{controlDict}

    \href{https://www.openfoam.com/documentation/user-guide/6-solving/6.1-time-and-data-inputoutput-control}{\blue{time control}}

    \href{https://www.cfd-online.com/Forums/openfoam-solving/128713-varying-time-step.html}{\blue{modify controlDict on the fly}}

    {\setstretch{1.0}
    \begin{minted}{c}
    /*application     rhoPimpleFoam;*/
    startFrom       latestTime;
    /*startFrom       startTime;*/
    startTime       0;
    stopAt          endTime;
    endTime         10;
    deltaT          5e-5;
    writeControl    adjustableRunTime;
    writeInterval   0.2;
    /*writeControl    timeStep;*/
    /*writeInterval   50;*/
    purgeWrite      0;
    writeFormat     ascii;
    /*writeFormat     binary;*/
    writePrecision  9;
    writeCompression off;
    timeFormat      fixed;
    timePrecision   6;
    runTimeModifiable true;
    adjustTimeStep  yes;
    maxCo           5.0;
    \end{minted}
    }

        \subsubsection{functions}
    {\setstretch{1.0}
    \begin{minted}{c}
    functions
    {
        ...
    }
    \end{minted}
    }

    \begin{minipage}{\textwidth}
    {\setstretch{1.0}
    \begin{minted}{c}
    gradient
    {
        type            grad;
        libs            ("libfieldFunctionObjects.so");
        field           T;

        // Optional (inherited) entries
        writePrecision  8;
        writeToFile     true;
        region          flow;
        writeControl    writeTime;
    }
    \end{minted}
    }
    \end{minipage}

    \begin{minipage}{\textwidth}
    \vspace{12pt}
    {\setstretch{1.0}
    \begin{minted}{c}
    wallHeatFlux
    {
        type            wallHeatFlux;
        libs            ("libfieldFunctionObjects.so");
        patches         ("solid_to_flow");
        qr              qr;
        writePrecision  8;
        writeToFile     true;
        region          flow;
        writeControl    writeTime;
    }
    \end{minted}
    }
    \begin{center}
    \fbox{\begin{minipage}[t]{0.8\textwidth}
    \centering
    PS: may have to modify the {\code constant/boundary} with wall patch. \\ If coupled domains, using {\code mappedWall}, it seems to work.
    \end{minipage}}
    \end{center}
    \end{minipage}

    \begin{minipage}{\textwidth}
    \vspace{12pt}
    {\setstretch{1.0}
    \begin{minted}{c}
    probes
    {
       type               probes;
       functionObjectLibs ("libsampling.so");
       writeControl       timeStep;
       writeInterval      5;
       region             flow;
       fields             (T);
       probeLocations (
           ( 0.0001 0.105 0.0)
           ...
       );
    }
    \end{minted}
    }
    \end{minipage}

    \begin{minipage}{\textwidth}
    \vspace{12pt}
    {\setstretch{1.0}
    \begin{minted}{c}
    patchProbes
    {
       type               patchProbes;
       functionObjectLibs ("libsampling.so");

       // Patches to sample (wildcards allowed)
       patchName          wall;

       // Name of the directory for probe data
       //name               patchProbes;

       writeControl       writeTime;
       //writeControl       timeStep;
       //writeInterval      5;

       //region             flow;
       fields             (wallHeatFlux);

        // Locations to probe. These get snapped onto the nearest point
        // on the selected patches
       probeLocations (
           ( 0.0 0.11 0.0)
       );
    }

    \end{minted}
    }
    \end{minipage}

    \begin{minipage}{\textwidth}
    \vspace{12pt}
    {\setstretch{1.0}
    \begin{minted}{c}
    line
    {
        type                sets;
        functionObjectLibs  ("libsampling.so");
        enabled             true;
        writeControl        timeStep;
        writeInterval       10;
        region              flow;
        interpolationScheme cellPoint;
        setFormat           raw;
        sets
        (
            line1
            {
                type lineUniform;
                axis distance;
                start ( 1e-5 0.105 0 );
                end   ( 0.05 0.105 0 );
                nPoints 10;
            }
        );
        fields ( p T );
    }
    \end{minted}
    }
    \end{minipage}

    \begin{minipage}{\textwidth}
    \vspace{12pt}
    {\setstretch{1.0}
    \begin{minted}{c}
    volAverage
    {
        libs             ("libfieldFunctionObjects.so");
        type             volFieldValue;
        operation        volAverage;
        region           porousMat;
        fields           (Ta tau rho_s[1] rho_s[2]);
        writeFields      false;
        writeControl     timeStep;
        writeInterval    5;
    }
    \end{minted}
    }
    \end{minipage}

    \begin{minipage}{\textwidth}
    \vspace{12pt}
    {\setstretch{1.0}
    \begin{minted}{c}
    volIntegrate
    {
        libs             ("libfieldFunctionObjects.so");
        type             volFieldValue;
        operation        volIntegrate;
        region           porousMat;
        fields           (rho_s[1] rho_s[2]);
        writeFields      false;
        writeControl     timeStep;
        writeInterval    5;
    }
    \end{minted}
    }
    \end{minipage}

    \subsection{setFields}

        Non-uniform IC, specified in {\code setFieldDict} inside {\code system/<region>}

        {\code setFields}

        {\code setFields -region fluid}

    \subsection{parallel}

        Define system/decomposeParDict
        \begin{minted}{c}
        numberOfSubdomains  <n>;
        method              scotch;
        \end{minted}
        to partition the mesh and IC, run: \\
        {\code decomposePar <-allRegions>}

        run openFOAM with mpi: \\
        {\code mpirun -np <n> <foamSolver> -parallel}

        Combine results with \\
        {\fontfamily{pcr}\selectfont reconstructPar <-allRegions>} \\
        where additional options can be used, such as: \\
          {\code -latestTime} :: select the latest time \\
          {\code -newTimes} :: only reconstruct new times (i.e. that do not exist already)

        \begin{center}
        \fbox{\begin{minipage}[t]{0.8\textwidth}
        \centering
        Depending on the version (7 vs 10), each sub-domain (fluid, wall, porous material etc) will need its own {\code decomposeParDict} in {\code system}
        \end{minipage}}
        \end{center}

        \begin{center}
        \fbox{\begin{minipage}[t]{0.8\textwidth}
        \centering
        To modify parameters in a parallel simulation, may be easier to recombine the latest solution file; modify then re-decompose
        \end{minipage}}
        \end{center}

    \subsection{changeDictionary}

    In parallel just use the "-parallel" option of changeDictionary, just as most OF tools work with cases.
    ( so the command line will be ``mpirun -np X changeDictionary -parallel'')

    \subsubsection{Erase entry}
    \begin{minipage}{\textwidth}
    \vspace{12pt}
    {\setstretch{1.0}
    \begin{minted}{c}
    U
    {
        boundaryField
        {
            ~fuel
            {
            }

            ~shield
            {
            }
        }
    }
    \end{minted}
    }
    \end{minipage}

    \subsubsection{Modify entry}

    Note that this will first modify the specific entries, then append new ones. Options not re-specified will be kept.

    \begin{minipage}{\textwidth}
    \vspace{12pt}
    {\setstretch{1.0}
    \begin{minted}{c}
    U
    {
        boundaryField
        {
            fuel
            {
                type            fixedValue;
                value           uniform (0 0.14517050853 0);
            }

            shield
            {
                type            fixedValue;
                value           uniform (0 0.14517050853 0);
            }
        }
    }
    \end{minted}
    }
    \end{minipage}

\clearpage
\section{Solvers}

    \subsection{rhoPimpleFoam}

        \href{https://www.openfoam.com/documentation/guides/latest/doc/guide-applications-solvers-compressible-rhoPimpleFoam.html}{\blue{link}}

    \subsection{chtMultiRegionFoam}

        Put all BCs in ``0/'' (including the wall as fluid\_to\_solid and solid\_to\_fluid)

        {\code splitMeshRegions -overwrite} plus: \\
          {\code -cellZones} :: the usual, where the domains will be split \\
          {\code -cellZonesOnly} :: in case one of the domains is disjoint, i.e., no shared nodes. Otherwise, it will be split.

        May need to copy a {\code regionProperties} to {\code constant}

        \href{https://www.cfd-online.com/Forums/openfoam/233208-conjugate-heat-transfer-contact-thermal-resistance-both-patches.html}{\blue{link}}

        \href{https://www.sciencedirect.com/science/article/pii/S1290072920312606?fr=RR-2&ref=pdf_download&rr=8b9465ed4a1ce157}{\blue{link}}

        \href{https://doc.openfoam.com/2306/tools/processing/solvers/rtm/heat-transfer/chtMultiRegionFoam/}{\blue{link}}

        \href{https://help.sim-flow.com/solvers/cht-multi-region-foam}{\blue{link}}

    \subsection{reactingFoam}

        \href{https://help.sim-flow.com/solvers/reacting-foam}{\blue{link}}

\clearpage
\section{postProcess}

    \href{https://doc.openfoam.com/2306/tools/post-processing/function-objects/#}{\blue{postProcessing}}

    {\code postProcess -func "grad(T)"}

    {\code postProcess -func writeCellVolumes -region porousMat}

    {\code postProcess -list}

    \subsection{foamToVTK}

        Useful in case paraFoam misbehaves (aka Segmentation Fault)

        If parallel simulation, first run {\code reconstructPar}.

        {\code foamToVTK}

        {\code -region flow} \\
        Older versions :: {\code -region $<$fluid$>$} then {\code -region $<$solid$>$} \\
        Newer versions :: {\code -allRegions}

        {\code -nearCellValue}

        {\code -fields '(p p\_rgh T U CO2 H2O CO N2)'}

        Specify initial time for conversion :: {\code -time 5.0:}

    \subsection{solver-dependent}

\clearpage
\section{Tutorials}

    \href{https://www.cfdyna.com/Home/of_Programming.html}{\blue{a link that looks very good}}

\clearpage
\section{Formulation}

    \subsection{Pressure}

        For buoyant solvers, pressure is computed with
        \begin{eqnarray}
        \rho g h + P' &=& P \\
        - \nabla(\rho g h) - \nabla P' &=& - \nabla P \\
        - \red{\cancel{\rho h \nabla g}} - \rho g \nabla h - g h \nabla \rho - \nabla P' &=& - \nabla P \\
        - \rho g - g h \nabla \rho - \nabla P' &=& - \nabla P
        \end{eqnarray}
        where $\nabla h = \mathbf{I}$.

        \href{
        https://www.cfd-online.com/Forums/openfoam-solving/219609-rho-buoyant-solver-p-p\_rgh-ph\_rgh.html
        }{\blue{link}}

        \href{
        https://www.openfoam.com/documentation/guides/latest/api/prghTotalHydrostaticPressureFvPatchScalarField\_8H\_source.html
        }{\blue{link}}

        \href{
        https://www.openfoam.com/documentation/guides/latest/api/classFoam\_1\_1prghTotalHydrostaticPressureFvPatchScalarField.html
        }{\blue{link}}

    \subsection{Energy equation}

        \begin{equation}
        C_P (T) = \sum_i c_i T^i = \frac{\partial e}{\partial T}
        \end{equation}

        \begin{equation}
        \kappa \nabla T = \kappa \frac{C_P}{C_P} \nabla T
        \end{equation}

        \begin{equation}
        \kappa \nabla T = \frac{\kappa}{C_P} \frac{\partial e}{\partial T} \nabla T
        \end{equation}

        \begin{equation}
        \kappa \nabla T = \frac{\kappa}{C_P} \nabla e
        \end{equation}

        \begin{equation}
        \kappa \nabla T = \alpha_\text{eff} \nabla e \quad \text{ where } \quad  \alpha_\text{eff} = \sfrac{\kappa}{C_p}
        \end{equation}

    \subsection{Species equation}

        See openFOAM-10 {\code unityLewisFourier.H}

        \begin{equation}
        Le = \frac{\alpha}{D}
        \end{equation}

        \begin{equation}
        Le = \frac{\kappa}{\rho C_p} \frac{1}{D} = \frac{\alpha_\text{eff}}{\rho D}
        \end{equation}

    {\setstretch{1.0}
    \begin{minted}{c}
        virtual tmp<scalarField> DEff
        (
            const volScalarField& Yi,
            const label patchi
        ) const
        {
            return
                this->thermo().kappa().boundaryField()[patchi]
               /this->thermo().Cp().boundaryField()[patchi];
        }
    \end{minted}
    }

    \subsection{Transport model}

        The individual species viscosity is given by
        \begin{equation}
            \mu_i = A_{s,i} \frac{\sqrt{T}}{1 + \frac{T_{s,i}}{T}}
        \end{equation}
        with mixture value given by weighted sum of individual viscosities by the respective mass fractions:
        \begin{equation}
            \mu = \sum_i Y_i \mu_i
        \end{equation}

        The thermal conductivity is given by the Eucken correlation as
        \begin{equation}
            \kappa = \mu c_v \left( 1.32 + 1.77 \frac{R}{c_v} \right)
        \end{equation}

        Species diffusivity based on unitary \textbf{Lewis} or \textbf{Schmidt} number (may depend on openFOAM version)

        \href{
        https://pure.tue.nl/ws/portalfiles/portal/305474394/1013373-Neural\_Network\_Hydrogen\_Combustion\_Modelling.pdf
        }{\blue{transport model}}

\end{document}